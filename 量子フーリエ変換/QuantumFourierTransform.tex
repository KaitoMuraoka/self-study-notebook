\documentclass[a4paper,11pt]{jsarticle}


% 数式
\usepackage{amsmath,amsfonts}
\usepackage{bm}
% 画像
\usepackage[dvipdfmx]{graphicx}
% ブラケット記法
\usepackage{braket}
% :=を実現させるため
\usepackage{mathtools}

\usepackage{qcircuit}


\begin{document}

\title{量子フーリエ変換}
\author{9BSP1118 村岡海人}
\date{\today}
\maketitle

\section{定義}
$2^n$成分の配列$\{x_j\}$に対して、その離散フーリエ変換である配列$\{y_k\}$を
\begin{equation}
  y_k \equiv  \frac{1}{\sqrt{2^n}}\sum ^{2^n -1} _{j = 0} x_j e^{i\frac{2\pi k j}{2^n}} \label{eq:hoge}
\end{equation}
で定義する。$(j = 0, \cdots , 2^n -1)$,$( k = 0, \cdots , 2^n -1 )$

また、量子フーリエ変換アルゴリズムは、入力の量子状態に、
\begin{equation*}
  \ket{x} \coloneqq \sum_{k = 0} ^{2^n -1} x_j \ket{j}
\end{equation*}
を
\begin{equation}
  \ket{y} \coloneqq \sum _{k = 0} ^{2^n -1} y_k \ket{k} \label{eq:fuga}
\end{equation}
となるように変換する量子アルゴリズムである。
% ここで、$\ket{i}$は、整数の2進数の表示$i_$
ここで、(\ref{eq:hoge})を(\ref{eq:fuga})に代入する。

\begin{eqnarray*}
  \ket{y} &=& \frac{1}{\sqrt{2^n}} \sum_{k = 0} ^{2^n - 1} \sum_{j = 0} ^{2^n -1} x_j e^{i \frac{2 \pi k j}{2^n}} \ket{k} \\
  &=& \sum_{j = 0}^{2^n -1} x_j \left( \frac{1}{\sqrt{2^n}} \sum_{k = 0}^{2^n - 1} e^{i \frac{2\pi k j}{2^n}} \ket{k} \right)
\end{eqnarray*}
となる。

よって、量子フーリエ変換は、
\begin{equation*}
  \ket{j} \rightarrow \frac{1}{\sqrt{2^n}} \sum _{k = 0} ^{2^n - 1} e^{i \frac{2\pi k j}{2^n}} \ket{k}
\end{equation*}
を行う量子回路を見つければよいことになる。

また、この式はさらに変形でき、$k = k_1 \cdot 2^{n-1} + \cdots + k_n \cdot 2^0$、$(k)_{10} = (k_1 k_2 \cdots k_n)_2$より、
\begin{eqnarray*} 
  \sum _{k = 0} ^{2^n - 1} e^{i \frac{2\pi k j}{2^n}} \ket{k} &=& \sum _{k_1 = 0} ^{1} \sum _{k_2 = 0} ^{1} \cdots \sum _{k_n = 0} ^{1} e^{i \frac{2\pi j(k_1 2^{n-1} + \cdots + k_n 2^0)}{2^n}} \ket{k_1 \cdots k_n} \\
  &=& \sum _{k_1 = 0} ^{1} \sum _{k_2 = 0} ^{1} \cdots \sum _{k_n = 0} ^{1} e^{i 2\pi j(k_1 2^{-1} + \cdots + k_n 2^{-n})} \ket{k_1 \cdots k_n} \\
  \text{
    因数分解して、全体をテンソル積で書き直すと、
    } \\
  &=& \left( \sum _{k_1 = 0} ^1 e^{2\pi j k_1 2^{-1}} \ket{k_1} \right) \otimes \cdots \otimes \left( \sum _{k_n = 0} ^1 e^{2\pi j k_n 2^{-n}} \ket{k_n} \right) \\
  % \text{
  %   10進数$j$は、2進数で、$j = j_1 \cdot 2^{n-1} + j_2 \cdot 2^{n-2} + \cdots + j_{n-1} \cdot 2 + j_n \cdot 2^0$なので、
  % } \\
  % \text{
  %   2進小数は、$0.j_1j_2\cdot j_{n-1}j_n = j_1 \cdot 2^{-1}$
  % }
\end{eqnarray*}
10進数$j$は、2進数で、
$$
j = j_1 \cdot 2^{n-1} + j_2 \cdot 2^{n-2} + \cdots + j_{n-1} \cdot 2 + j_n \cdot 2^0
$$
なので、2進小数は、
$$
0.j_1j_2\cdots j_{n-1}j_n = j_1 \cdot 2^{-1} + j_2 \cdot 2^{-2} + \cdots + j_{n-1} \cdot 2^{-n+1} + j_n \cdot 2^{-n}
$$
となるから、これを用いて、括弧内を計算すると、
\begin{eqnarray*}
  \sum _{k = 0} ^{2^n - 1} e^{i \frac{2\pi k j}{2^n}} \ket{k} = \left( \ket{0} + e^{2\pi 0.j_n} \ket{1} \right) \otimes \left( \ket{0} + e^{2\pi 0.j_{n-1} j_n} \ket{1} \right) \otimes \cdots \otimes \left( \ket{0} + e^{2\pi 0.j_1j_2 \cdots j_n} \ket{1} \right)
\end{eqnarray*}
となる。

まとめると、量子フーリエ変換は、
\begin{eqnarray}
  \ket{j} = \ket{j_1 \cdots j_n} \to \frac{\left( \ket{0} + e^{2\pi 0.j_n} \ket{1} \right) \otimes \cdots \otimes \left( \ket{0} + e^{2\pi 0.j_1j_2 \cdots j_n} \ket{1} \right)}{\sqrt{2^n}} 
  \label{eq: goal}
\end{eqnarray}
という変換ができればよい。

\section{回路の作成}
量子フーリエ変換を実行する回路を作成していく。
そのため、アダマールゲート$H$についての等式、
\begin{equation*}
  H \ket{M} = \frac{\ket{0} + e^{i 2\pi 0.m} \ket{1}}{\sqrt{2}} \qquad (m = 0, 1)
\end{equation*}
と、角度$\tfrac{2 \pi}{2^l}$の一般位相ゲート
\begin{eqnarray*}
  Re = 
  \begin{pmatrix}
    1 & 0 \\
    0 & e^{i \frac{2\pi}{2^l}}
  \end{pmatrix}
\end{eqnarray*}
を利用する。

まず初めに、状態$(\ket{0} + e^{i 2 \pi 0.j_1j_2\cdots j_n} \ket{1})$の部分を作る。
1番目の量子ビット$\ket{j_1}$にアダマールゲートを作用させると、
\begin{eqnarray*}
  \ket{j_1\cdots j_n} \to \frac{1}{\sqrt{2}} (\ket{0} + e^{i 2\pi 0.j_1} \ket{1}) \ket{j_2 \cdots j_n}
\end{eqnarray*}
となる。ここで、2番目のビット$\ket{j_2}$を制御ビットとする一般位相ゲート$R_2$を1番目の量子ビットにかけると、$j_2 = 0$の時は何もせず、$j_2 = 1$の時は1番目の量子ビットの$\ket{1}$部分に位相$2\pi / 2^2 = 0.01$(2進小数)がつくため、
\begin{eqnarray*}
  \frac{1}{\sqrt{2}} (\ket{0} + e^{i 2\pi 0.j_1} \ket{1}) \ket{j_2 \cdots j_n} \to \frac{1}{\sqrt{2}} (\ket{0} + e^{i 2\pi 0.j_1 j_2} \ket{1}) \ket{j_2 \cdots j_n}
\end{eqnarray*}
となる。以下、l番目の量子ビット$\ket{j_l}$を制御ビットとする一般位相ゲート$R_l$をかけると$(l = 3, \cdots, n)$、最終的に、
\begin{eqnarray*}
  \frac{1}{\sqrt{2}}(\ket{0} + e^{i 2\pi 0.j_1 \cdots j_n} \ket{1}) \ket{j_2 \cdots j_n}
\end{eqnarray*}
が得られる。

次に、状態$(\ket{0} + e^{i 2\pi 0.j_2 \cdots j_n} \ket{1})$の部分を作る。先ほどと同様に、2番目のビット$\ket{j_2}$にアダマールゲートを作用させると、
\begin{eqnarray*}
  \frac{1}{\sqrt{2}}(\ket{0} + e^{i 2\pi 0.j_1 \cdots j_n} \ket{1}) \frac{1}{\sqrt{2}} (\ket{0} + e^{i 2\pi 0.j_2} \ket{1}) \ket{j_3 \cdots j_n}
\end{eqnarray*}
となる。再び、3番目の量子ビットを制御ビット$\ket{j_3}$とする位相ゲート$R_2$をかけると、
\begin{eqnarray*}
  \frac{1}{\sqrt{2}}(\ket{0} + e^{i 2 \pi 0.j_1 \cdots j_n}\ket{1}) \frac{1}{\sqrt{2}} (\ket{0} + e^{i 2\pi 0.j_2 j_3} \ket{1}) \ket{j_3 \cdots j_n}
\end{eqnarray*}
となる。これを繰り返して、
\begin{eqnarray*}
  \frac{1}{\sqrt{2}}(\ket{0} + e^{i 2 \pi 0.j_1 \cdots j_n}\ket{1}) \frac{1}{\sqrt{2}} (\ket{0} + e^{i 2\pi 0.j_2 \cdots j_n} \ket{1}) \ket{j_3 \cdots j_n}
\end{eqnarray*}
となる。

以上のことを、$l$番目の量子ビット$\ket{j_l}$にアダマールゲート・制御位相ゲート$R_l, R_{l+1}, \cdots$をかけていく$(l = 3, \cdots , n)$。
すると、最終的に、
\begin{eqnarray*}
  \ket{j_1 \cdots j_n} \to \left( \frac{\ket{0} + e^{i 2\pi 0.j_1 \cdots j_n}\ket{1}}{\sqrt{2}} \right) \otimes \left( \frac{\ket{0} + e^{i 2\pi 0.j_2 \cdots j_n}\ket{1}}{\sqrt{2}} \right) \otimes \cdots \otimes \left( \frac{\ket{0} + e^{i 2\pi 0.j_n}\ket{1}}{\sqrt{2}} \right)
\end{eqnarray*}
となる。

(\refeq{eq: goal})と比較すると、ビットの順番が逆になっているので、SWAPゲートでビットの順番を反転させてあげれば、量子フーリエ変換を実行する回路が構成でいたことになる。

SWAPゲートを除いた、量子回路図で書くと、図\refeq{fig: quantumCircuit}となる。

\begin{figure}[htbp]
  \centering
  \includegraphics[width=15cm]{img/quantumCircuit.png}
  \caption{SWAPゲートを除いた量子フーリエ変換の量子回路}
  \label{fig: quantumCircuit}
\end{figure}  

\end{document}