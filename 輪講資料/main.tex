\RequirePackage{plautopatch}
\RequirePackage[l2tabu, orthodox]{nag}

\documentclass[platex,dvipdfmx]{jlreq}			% for platex
% \documentclass[uplatex,dvipdfmx]{jlreq}		% for uplatex
\usepackage{graphicx}
\usepackage{bxtexlogo}
\usepackage{braket}
\usepackage{amsthm}
\newtheorem{dfn}{定義}
\newtheorem{thm}{公理}
\newtheorem*{thm3}{公理2'}
\usepackage{amsmath}
\usepackage{mathtools}


\title{2.2量子力学の公理}

\author{9BSP1118 村岡海人}
\date{6月27日}
\begin{document}
\maketitle
\section*{2.2.1 状態空間}
% 量子力学の最初の公理によって量子力学の活動の舞台が設定される.その舞台は線形代数で身近になったHilbert空間である.
量子力学の最初の公理によって量子力学を記述するための概念が設定される。それが、Hilbert空間である。
\begin{thm}
    任意の孤立した物理系に関して、系の\textbf{状態空間}と呼ぶ内積を伴う複素ベクトル空間(つまりHilbert空間)が存在する.系は状態空間の単位ベクトルである\textbf{状態ベクトル}によって完全に記述できる.
\end{thm}

最も簡単で我々が最も関心を持つ量子力学の系は\textbf{qビット}である。qビットは2次元状態空間の元である。$\ket{0}$と$\ket{1}$がその状態空間の基底を形成する。そのとき状態空間における任意の状態ベクトルは次のように書ける。
\begin{equation}
    \ket{\psi} = a\ket{0} + b\ket{1} \tag{2.82}
\end{equation}
% \begin{align*}
%     \ket{\psi} = a\ket{0} + b\ket{1} \tag{foo} \\
%     \ket{0}  =
% \end{align*}
ここで、$a$と$b$は複素数である。$\ket{\psi}$が単位ベクトルであるという条件$\braket{\psi | \psi} = 1$は$|a|^2 + |b|^2 = 1$と等価であり、条件$\braket{\psi | \psi} = 1$を状態ベクトルの\textbf{正規化条件}と呼ぶ。


qビットについて議論するとき、正規直交の基底ベクトル集合$\ket{0}$と$\ket{1}$を常に参照し、その規定は前もって固定していると考える。直感的には状態$\ket{0}$と$\ket{1}$はビットがとる2つの値$0$と$1$に相似している。qビットがビットと異なるのは2つの状態の\textbf{重ね合わせ}$a\ket{0} + b\ket{1}$が存在し得ることであり、その場合はqビットが確定的に$\ket{0}$状態にあるとか確定的に$\ket{1}$状態にあるといえない。


任意の線型結合$\sum_i \alpha_i\ket{\psi_i}$は状態$\ket{\psi_i}$に対する\textbf{振幅}$\alpha_i$を状態$\ket{\psi_i}$を以て重ねたものである。したがって、例えば、
\begin{equation*}
    \frac{\ket{0} - \ket{1}}{\sqrt{2}} \tag{2.83}
\end{equation*}
は状態$\ket{0}$に対して振幅$1/\sqrt{2}$、状態$\ket{1}$に対して振幅$-1/\sqrt{2}$で、状態$\ket{0}$と$\ket{1}$を重ね合わせたものである。

\section*{2.2.2 時間発展}
\begin{thm}
    \textbf{閉じた}量子系の時間発展は\textbf{ユニタリー変換}で記述される。つまり、時刻$t_1$における系の状態$\ket{\psi}$は、時刻$t_2$における系の状態$\ket{\psi'}$と時刻$t_1$と$t_2$だけに依存するユニタリーオペレータ$U$によって関係付けられる。
    \begin{equation*}
        \ket{\psi'} = U \ket{\psi} \tag{2.84}
    \end{equation*}
\end{thm}
% 量子力学が特定の量子システムの状態空間や量子状態を教えてくれないのと同様に、実世界の量子ダイナミックスを記述するユニタリーオペレータ$U$がどのようなものか教えてくれない。
量子計算と量子情報において重要な、単一qビットに関するユニタリーオペレータの例を2、3見てみる。
すでにそのようなユニタリーオペレータの例を2.2.3項でPauliの行列を定義した。また、古典NOTゲートのアナロジーから$X$行列は量子NOTゲートであることが知られている。
PauliのXとZ行列はそれぞれ\textbf{ビット反転}および\textbf{位相反転}行列とも呼ばれている。

もう1つのユニタリーオペレータは\textbf{Hadamardゲート}であり、$H$と表記する。このオペレータの作用は$H\ket{0} \equiv (\ket{0} + \ket{1})/\sqrt{2}$, $H\ket{0} \equiv (\ket{0} - \ket{1})/\sqrt{2}$ で定義され、行列表現では次のように与えられる:
\begin{equation*}
    H = \frac{1}{\sqrt{2}}
    \begin{bmatrix}
        1 & 1 \\
        1 & -1 \tag{2.85}
    \end{bmatrix}
\end{equation*}

公理2は2つの異なる時刻における閉じた量子状態がどのように関係するかについて述べている。この公理を発展させると、量子系の進化を連続時間で記述できる。
% この発展した公理から公理2を回復することもできる
発展した公理を述べる前に以下、2つのことを述べる。
\begin{itemize}
    \item  表記状の注意:次の議論に現れる$H$はHadamardオペレータと同じではない
\end{itemize}

\begin{thm3}
    閉じた量子系の状態の時間発展は\textbf{Schr\"{o}dinger}方程式
    \begin{equation*}
        i\hbar \frac{d\ket{\psi}}{dt} = H\ket{\psi} \tag{2.86}
    \end{equation*}
    で記述される。$\hbar$は\textbf{Planck}の定数。$H$はHermiteオペレータであり、閉じた系のハミルトニアンと呼ぶ。
\end{thm3}

HamiltonianはHermiteオペレータなので、固有値$E$とそれに対応する正規化固有ベクトル$\ket{E}$を用いてスペクトル分解できる。
\begin{equation*}
    H = \sum _E E \ket{E}\bra{E} \tag{2.87}
\end{equation*}
状態$\ket{E}$は通常\textbf{エネルギー固有状態}または\textbf{定常状態}とも呼ばれ、$E$は状態$\ket{E}$の\textbf{固有エネルギー}である。最も低いエネルギーは系の\textbf{基底エネルギー}といい、対応するエネルギー固有状態を\textbf{基底状態}という。
$\ket{E}$を定常状態と呼ぶのは、唯一の時間変化が全体にかかる位相因子にのみ現れるのが理由である。
\begin{equation*}
    \ket{E} \rightarrow \exp(- i E t / \hbar)\ket{E} \tag{2.88}
\end{equation*}
例えば、単一のqビットのHamiltonianがPauli行列のみを用いて、
\begin{equation*}
    H = \hbar \omega X \label{2.89} \tag{2.89}
\end{equation*}
で与えられたとする。
このHamiltonianのエネルギー固有状態は明らかに$X$の固有状態と同じ、状態が$(\ket{0} + \ket{1})/\sqrt{2}$と$(\ket{0} - \ket{1})/ \sqrt{2}$、それぞれに対応するエネルギーが$\hbar \omega$と$- \hbar \omega$で与えられる。したがって、基底状態は$(\ket{0} - \ket{1})/ \sqrt{2}$、基底状態エネルギーは$- \hbar \omega$である。 
ダイナミックスを記述するHamiltonian描像の公理$2'$とユニタリーオペレータ描像の公理$2$の関係は、Schr\"{o}dinger方程式の解を書き下すと得られ、次式が容易に証明できる:
\begin{equation*}
    \ket{\psi(t_2)} = \exp\left[ \frac{-i H (t_2 - t_1)}{\hbar} \right]\ket{\psi(t_1)} = U(t_1, t_2)\ket{\psi(t_1)}  \tag{2.90}
\end{equation*}
ここで、
\begin{equation*}
    U(t_1, t_2) \equiv \exp\left[ \frac{-i H (t_2 - t_1)}{\hbar} \right] \tag{2.91}
\end{equation*}
と定義する。
\end{document}